\documentclass[10pt,a4j]{jarticle}
\usepackage[dvipdfmx]{graphicx}
\usepackage[dvipdfmx]{color}
\usepackage{url}
\usepackage{here}
\usepackage{amsmath}
\usepackage{subfig}

%\newcommand{\figref}[1]{図\ref{#1}}
\newcommand{\subfigref}[2]{図\ref{#1}\subref{#2}}
%\renewcommand{\eqref}[1]{式(\ref{#1})}



\renewcommand{\topfraction}{1.0}
\renewcommand{\bottomfraction}{1.0}
\renewcommand{\dbltopfraction}{1.0}
\renewcommand{\textfraction}{0.01}
\renewcommand{\floatpagefraction}{1.0}
\renewcommand{\dblfloatpagefraction}{1.0}
\setcounter{topnumber}{5}
\setcounter{bottomnumber}{5}
\setcounter{totalnumber}{10}

%----------表紙設定----------%
\makeatletter
\def\id#1{\def\@id{#1}}
\def\department#1{\def\@department{#1}}
\def\faculty#1{\def\@faculty{#1}}
\def\adviser#1{\def\@adviser{#1}}
\def\@maketitle{
\begin{center}
{\Large \@title \par} % 論文のタイトル部分
\vspace{40mm}
{\large \@date\par} % 提出年月日部分
\vspace{15mm}
{\large \@faculty \par} % 学部
\vspace{15mm}
{\large \@department \par} % 所属部分
\vspace{50mm}
{\large \@id \par} % 学籍番号
\vspace{5mm}
{\large \@author \par} % 氏名
\vspace{5mm}
{\large 指導教員:\ \@adviser \par} % 指導教員
\vspace{10mm}
{\large 豊橋技術科学大学 \par} %大学名
\end{center}
\par\vskip 1.5em
}
\makeatother

\date{\today} % 日付
\faculty{学部(工学)} % 学部
\department{情報・知能工学課程} % 所属課程
\id{153321} % 学籍番号
\author{加藤\ 翼} % 氏名
\adviser{三浦\ 純} % 指導教員

\begin{document}
%---表紙---%
%	\maketitle
%	\thispagestyle{empty}
%	\setcounter{page}{0}
%	\newpage
	
%---概要---%
\newcommand{\figref}[1]{図\ref{#1}}
\renewcommand{\eqref}[1]{式(\ref{#1})}
\newcommand{\xia}{\xi_\alpha}
\newcommand{\barxia}{\bar{\xi_a}}
\subsection{最尤推定で楕円のパラメータを推定するときの目的関数が以下のようになることを示せ}
\begin{equation}
 J_{ML} = \sum_{\alpha=1}^N \frac{(u,\xi_\alpha)^2}{(u , V_0[\xi_\alpha]u)}
\end{equation}
ただし,\begin{math}V_0[\xi_a]はデータベクトル\xi_a\end{math}に関する正規化共分散行列を表す.\\ \\

\begin{math}
\xi 空間においてノイズモデルを正規分布と仮定すると,最尤推定は次のマハラノビス距離の二乗和の最小化となる.また,ここで
\xi_\alpha の真値を\bar{\xi_\alpha}とおくと,以下の式で表すことができる.
\end{math}

\begin{equation}
 J = \sum_{\alpha=1}^N (\xia - \barxia , V[\xia]^{-1}(\xia - \barxia) )
\end{equation}
これを拘束条件
\begin{equation}
 \label{eq:rule}
  (\barxia,u) = 0 \ \ ,\  \alpha = 1, ... , N 
\end{equation}
のもとで最小となる
\begin{math}
\barxia , u を考える.\eqref{eq:rule}は\barxia に関して線形であるため,ラグランジュ乗数を用いて削除することができる.
\end{math}

%\begin{thebibliography}{9}
%	\fontsize{8.5pt}{0pt}\selectfont
%	\bibitem{bunken1}{藤吉研究室,"画像局所特徴量と特定物体認識",\url{http://www.vision.cs.chubu.ac.jp/cvtutorial/PDF/02SIFTandMore.pdf}}
%\end{thebibliography}	


\end{document}