\newcommand{\figref}[1]{図\ref{#1}}
\renewcommand{\eqref}[1]{式(\ref{#1})}
\newcommand{\xia}{\xi_\alpha}
\newcommand{\barxia}{\bar{\xi_a}}
\subsection{最尤推定で楕円のパラメータを推定するときの目的関数が以下のようになることを示せ}
\begin{equation}
 J_{ML} = \sum_{\alpha=1}^N \frac{(u,\xi_\alpha)^2}{(u , V_0[\xi_\alpha]u)}
\end{equation}
ただし,\begin{math}V_0[\xi_a]はデータベクトル\xi_a\end{math}に関する正規化共分散行列を表す.\\ \\

\begin{math}
\xi 空間においてノイズモデルを正規分布と仮定すると,最尤推定は次のマハラノビス距離の二乗和の最小化となる.また,ここで
\xi_\alpha の真値を\bar{\xi_\alpha}とおくと,以下の式で表すことができる.
\end{math}

\begin{equation}
 J = \sum_{\alpha=1}^N (\xia - \barxia , V[\xia]^{-1}(\xia - \barxia) )
\end{equation}
これを拘束条件
\begin{equation}
 \label{eq:rule}
  (\barxia,u) = 0 \ \ ,\  \alpha = 1, ... , N 
\end{equation}
のもとで最小となる
\begin{math}
\barxia , u を考える.\eqref{eq:rule}は\barxia に関して線形であるため,ラグランジュ乗数を用いて削除することができる.
\end{math}
